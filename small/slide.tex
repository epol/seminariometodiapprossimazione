\documentclass{beamer}

\usetheme{metropolis}

\metroset{sectionpage=progressbar,progressbar=frametitle}
\metroset{numbering=fraction}
\metroset{block=fill}

\usepackage{pgfpages}
%\setbeameroption{show notes on second screen}

\usepackage{appendixnumberbeamer}

\usepackage[utf8x]{inputenc}
\usepackage[english]{babel}

\usepackage{amsmath}
\usepackage{amssymb}
\usepackage{amsthm}
\usepackage{xfrac}
\usepackage{mathtools}
\usepackage{graphicx}
\usepackage{hyperref}
\usepackage{mathtools}

% Plot e disegnini
\usepackage{pgfplots}
\pgfplotsset{compat=1.14}


\newcounter{counter1}

\theoremstyle{plain}
\newtheorem{myteo}[counter1]{Theorem}
\newtheorem{mylem}[counter1]{Lemma}
\newtheorem{mypro}[counter1]{Proposition}
\newtheorem{mycor}[counter1]{Corollary}
%\newtheorem*{myteo*}{Teorema}
%\newtheorem*{mylem*}{Lemma}
%\newtheorem*{mypro*}{Proposizione}
%\newtheorem*{mycor*}{Corollario}

\theoremstyle{definition}
\newtheorem{mydef}[counter1]{Definition}
\newtheorem{myes}[counter1]{Example}
%\newtheorem{myex}[counter1]{Esercizio}
%\newtheorem*{mydef*}{Definizione}
%\newtheorem*{myes*}{Esempio}
%\newtheorem*{myex*}{Esercizio}

\theoremstyle{remark}
%\newtheorem{mynot}[counter1]{Nota}
\newtheorem{myoss}[counter1]{Remark}
%\newtheorem*{mynot*}{Nota}
%\newtheorem*{myoss*}{Osservazione}

\newcommand{\obar}[1]{\overline{#1}}
\newcommand{\ubar}[1]{\underline{#1}}

\newcommand{\set}[1]{\left\{#1\right\}}
\newcommand{\pa}[1]{\left(#1\right)}
\newcommand{\ang}[1]{\left<#1\right>}
\newcommand{\bra}[1]{\left[#1\right]}
\newcommand{\abs}[1]{\left|#1\right|}
\newcommand{\norm}[1]{\left\|#1\right\|}
\newcommand{\ceil}[1]{\left\lceil#1\right\rceil}
\newcommand{\floor}[1]{\left\lfloor#1\right\rfloor}

\newcommand{\pfrac}[2]{\pa{\frac{#1}{#2}}}
\newcommand{\bfrac}[2]{\bra{\frac{#1}{#2}}}
\newcommand{\psfrac}[2]{\pa{\sfrac{#1}{#2}}}
\newcommand{\bsfrac}[2]{\bra{\sfrac{#1}{#2}}}

\newcommand{\der}[2]{\frac{\partial #1}{\partial #2}}
\newcommand{\pder}[2]{\pfrac{\partial #1}{\partial #2}}
\newcommand{\sder}[2]{\sfrac{\partial #1}{\partial #2}}
\newcommand{\psder}[2]{\psfrac{\partial #1}{\partial #2}}

\newcommand{\intl}{\int \limits}

\DeclareMathOperator{\de}{d}
\DeclareMathOperator{\id}{Id}
\DeclareMathOperator{\len}{len}

\DeclareMathOperator{\gl}{GL}
\DeclareMathOperator{\aff}{Aff}
\DeclareMathOperator{\isom}{Isom}

\DeclareMathOperator{\im}{Im}
\DeclareMathOperator{\re}{Re}
\DeclareMathOperator{\sign}{sign}




\title{Altre iterazioni per la radice quadrata di matrici}
\date{\today}
\author{Enrico Polesel}
%\institute{Universit\`a di Pisa}


\begin{document}
\maketitle

\section{Definition and direct method}

\begin{frame}{Principal square root}
  \begin{columns}[T]
    \begin{column}{.45\textwidth}
      For $z\in \mathbb{C}$ the equation $y^2 = z$ has two solutions $y_1,
      y_2$. It's easy to verify that $y_1 = -y_2$.
      \vspace{20pt}
      
      We call \textit{principal squre root} a solution such that
      $\re\pa{y} > 0$. Such $y$ exists (and it's unique) when $z
      \not\in \mathbb{R}^-$.
      \vspace{20pt}

      We write $y = z ^{1/2}$ or $y = \sqrt{z}$.
    \end{column}
    \begin{column}{.6\textwidth}
      \begin{figure}
        \begin{tikzpicture}
          \begin{axis}[axis lines=middle,axis equal,xmin=-3.5,xmax=2.5]
            \addplot [blue,data cs=polar] coordinates{(0,0) (140,4)};
            \node at (axis cs:-3.2,2.4) {$z$};
            \addplot [red,data cs=polar] coordinates{(0,0) (70,2)};
            \addplot [red,data cs=polar] coordinates{(0,0) (250,2)};
            \node at (axis cs:1,2) {$y_1$};
            \node at (axis cs:-1,-2) {$y_2$};
          \end{axis}
        \end{tikzpicture}  
      \end{figure}
    \end{column}
  \end{columns}
\end{frame}

\begin{frame}{Definition}
  \begin{myteo}
    Let $A \in \mathbb{C} ^ {n \times n}$ have no eigenvalues on
    $\mathbb{R}^-$. There is a unique square root $X$ of $A$ all of
    whose eigenvalues lay in the open right half-plane, and it is a
    primary matrix function of $A$. 
  \end{myteo}

  We refer to $X$ as the \textit{principal square root} of $A$ and
  write $X = A^{1/2}$.
  \vfill

  \begin{mypro}
    If $A$ is real then $A^{1/2}$ is real.    
  \end{mypro}
\end{frame}

\begin{frame}{Schur Method}
  Let $A \in \mathbb{C}^{n\times n}$ be a nonsingolar matrix without
  eigenvalues in $\mathbb{R}^-$, given a Schur decomposition
  $A = QTQ^*$ we compute $A^{1/2} = Q U Q^*$ with $U$ such that:
  \begin{align*}
    u_{ii}^2 &= t_ii \\
    \pa{u_{ii} + u_{jj}} u_{ij} &= t_{ij} - \sum _{k=i+1} ^{j-1}
                                  u_{ik}u_{kj}
  \end{align*}
  
  \begin{block}{Cost}
    $\pa{ 28+\frac{1}{3}} n^3$ floating point operations.
  \end{block}
\end{frame}

\begin{frame}{Variations of Schur method}
  The Schur method can be adapted to support singular matrices when
  $0$ is a semisimple eigenvalue.
  \[ T = \bra{
      \begin{matrix}
        T_{11} & T_{12} \\
        0 & 0 
      \end{matrix}
    } \]
  \vfill

  When $A \in \mathbb{R}^-$ we can use the real Schur decomosition to
  avoid complex arithmetic (with the same cost).
\end{frame}

\begin{frame}{Stability of the Schur method}
  Let $\hat {X}$ be the computed approximation of $X$, then
  \[ \norm{{\hat{X}}^2 - A} _F \le \gamma _n \norm{\hat{X}}_F ^2 \]
  
  TODO: è quella ottima
\end{frame}

\section{Iterative methods}

\subsection{Newton iteration}

\begin{frame}{Newton's method derivation}
  Let $Y$ be an approximate solution of the equation $X^2 = A$, so let
  $X = Y+E$, then:
  \[ A = \pa{Y+E}^2 = Y^2 + YE + EY + E^2 \]
  If we omit the second order in $E$ and we fix the first
  approximation $X_0$ we obtain the following iteration:
  \begin{align*}
    X_k E_k + E_k X_k &= A - X_k ^2 \\
    X_{k+1} &= X_k + E_k
  \end{align*}
  
  Solving the Sylvester equation for $E_k$ at every step we obtain a
  succession of matrices $\pa{ X_k} _{k\in \mathbb{N}}$.
\end{frame}

\begin{frame}
  \begin{align*}
    X_k E_k + E_k X_k &= A - X_k ^2 \\
    X_{k+1} &= X_k + E_k
  \end{align*}
  This Sylvester equation is non singular if and only if $X_k$ and
  $-X_k$ have no eigenvalues in common.
  \vfill

  \begin{myoss}
    We need to resolve a Sylvester equation for each iteration, so we
    pay at least $25n^3$ flops (a Schur decomposition) every time!
  \end{myoss}
\end{frame}

\begin{frame}
  If $X_k$ commutes with $E_k$ we can write
  \[ E_k = \frac{1}{2} X_k ^{-1} \pa{ A - X_k ^2} \Rightarrow X_{k+1}
    =  \frac{1}{2} \pa{ X_k ^{-1} A + X_k } \]
  \pause
  \begin{mylem}
    Suppose that in the Newton iteration $X_0$ commutes with $A$ and
    all the iterates are well-defined. Then, for all $k$, $X_k$
    commutes with $A$ and
    \[ X_{k+1} = \frac{1}{2} \pa{ X_k + X_k ^{-1} A} \]
  \end{mylem}
  \begin{proof}
    By induction.
  \end{proof}
\end{frame}

\begin{frame}{Newton's method}
  \begin{block}{Newton iteration}
    Setting $X_0 = A$ (or $X_0 = I$) we obtain
    \[ \left\{
        \begin{matrix}
          X_0 &=& A \\
          X_{k+1} &=& \frac{1}{2} \pa{ X_k + X_k ^{-1} A}
        \end{matrix} \right.
    \]
  \end{block}

  If $A$ is nonsingular it can be proved that this method has
  quadratic convergence for $X_0$ chosen sufficiently close to
  $A^{1/2}$.
\end{frame}

\begin{frame}{Newton's method convergence}
  \begin{myteo}
    Let $A \in \mathbb{C}^{n\times n}$ have no eigenvaules on
    $\mathbb{R}^-$. The Newton iterates $X_k$ with any $X_0$ that
    commutes with $A$ are related to the Newton sign iterates:
    \begin{align*}
      S_{k+1} = & \frac{1}{2} \pa{ S_k + S_k^{-1}} \\
      S_0 = & A^{-1/2} X_0 
    \end{align*}
    by $X_k = A^{1/2} S_k$.
  \end{myteo}
  \begin{proof}
    By induction on $k$:
    \[ X_{k+1} = \frac{1}{2}\pa{A^{1/2}S_k + S_k^{-1}A^{-1/2}A} =
        A^{1/2}\frac{1}{2}\pa{S_k + S_k ^{-1}} \]
  \end{proof}
\end{frame}

\begin{frame}{Newton's method convergence}
  \begin{myteo}
    If $A^{-1/2}X_0$ has no pure imaginary eigenvaules, the $X_k$
    are defined and $X_k$ converges quadratically to
    $A^{1/2}\sign\pa{A^{-1/2} X_0}$.

    If the spectrum  of $A^{-1/2} X_0$ lies in the right half-plane
    then $X_k$ converges quadratically to $A^{1/2}$ and, for any
    consistent norm,
    \[ \norm{X_{k+1} - A^{1/2}} \le \frac{1}{2}\norm{X_k ^{-1}}
      \norm{X_k - A^{1/2}} ^2 \]
  \end{myteo}
  \vfill
  
  The hypothesis that $A$ has no eigenvaules in $\mathbb{R}^-$ and our
  choice of the principal square root function imply that the spectrum
  of $A^{1/2}$ is contained in the open right half-plane.
\end{frame}

\begin{frame}{Newton's method speed of convergence}
  We have proved that the Newton's method converges for any choice of
  $X_0$ that commutes with $A$, moreover it converges to $A^{1/2}$
  quadratically for $X_0 = A$.
  \vfill
  
  It's known that the Newton iteration for $\sign\pa{M}$ requires many
  iterations when $M$ has an eigenvalue close to the imaginary axis.

  Recalling the relation $X_k = A^{1/2} S_k$ we can conclude that our
  Newton iteration requires many iterations when $A$ has an eigenvalue
  close to the negative real axis.  
\end{frame}

\subsection{DB iteration}

\begin{frame}{Denman and Beavers (DB) iteration}
  If we set $Y_k = A^{-1} X_k$ we can write
  \[ X_{k+1} = \frac{1}{2}\pa{X_k + Y_k ^{-1}} \]
  \[ Y_{k+1} = A^{-1} X_{k+1} = \frac{1}{2}\pa{ Y_k + X_k^{-1}} \]
  (using, again, the fact that $X_k$ commutes with $A$)

  \begin{block}{DB iteration}
    \begin{align*}
      X_{k+1} &= \frac{1}{2} \pa{ X_k + Y_k ^{-1}} & X_0 &= A \\
      Y_{k+1} &= \frac{1}{2} \pa{ Y_k + X_k ^{-1}} & Y_0 &= I 
    \end{align*}
  \end{block}
\end{frame}

\begin{frame}{DB convergence}
  The DB iteration is an algebraic manipulation of the Newton
  iteration, so if $A$ has no eigenvalues on $\mathbb{R}^-$ then we
  can write:
  \begin{align*}
    \lim _{k \to \infty} X_k &= A^{1/2} \\
    \lim _{k\to \infty} Y_k &= A^{-1/2}
  \end{align*}  
  It has the same proprieties of the Newton iterations.

\end{frame}


\end{document}